% !TEX TS-program = xelatex
% !TEX encoding = UTF-8 Unicode
% !Mode:: "TeX:UTF-8"

\documentclass{resume}
\usepackage{zh_CN-Adobefonts_external} % Simplified Chinese Support using external fonts (。/fonts/zh_CN-Adobe/)
%\usepackage{zh_CN-Adobefonts_internal} % Simplified Chinese Support using system fonts
\usepackage{linespacing_fix} % disable extra space before next section
\usepackage{cite}
\usepackage{hyperref}

\begin{document}
\pagenumbering{gobble} % suppress displaying page number

\name{golang后台研发-赵轩超}

% {E-mail}{mobilephone}{homepage}
% be careful of _ in emaill address
\contactInfo{(+86) 177-1164-4027}{ffzxc.do@gmail.com}{GitHub @zput}{}
% {E-mail}{mobilephone}
% keep the last empty braces!
%\contactInfo{xxx@yuanbin。me}{(+86) 131-221-87xxx}{}
 
\section{个人信息}
\begin{itemize}[parsep=0.2ex]
  \item 赵轩超/男/1993-12
  \item 本科/湖南理工学院/自动化(2012.09-2016.06)
  \item 个人技术博客:\url{https://zput.github.io}
  \item Github:\url{https://github.com/zput}
\end{itemize}

\section{工作经历}

\datedsubsection{\textbf{深圳乐天童创科技有限公司}, 平台架构部golang高级开发工程师}{2019.7-至今}
\begin{itemize}[parsep=0.2ex]
   \item 企业上云,整合服务,上k8s。整理前存在十几个beego独立的微服务,逻辑混乱,前端调用的接口URI有相同前缀但不同后缀,各自分布在多个微服务中,无法进行负载均衡,扩容分散流量,作为发起人,重构服务,商讨路由整合, 缩减微服务到7个。并编写helm, 让服务都上到k8s。性能从10QPS提升到2000QPS,测试加正式服务器由10台减少到5台。
      %(\url(helm参考)
      % - 2019-10-24 22:03:00

   \item 规划CI/CD流程,基于gitlab's CI/CD自动发布k8s。包括.gitlab-ci.yml,编译/发布使用的Dockerfile文件编写和gitlab-runner部署等。当自动发布上线后,提交代码打包部署到交给测试人员测试,从至少15分钟,降低到5分钟内。
    \begin{itemize}[parsep=0.1ex]
      \item \url{https://zput.github.io/2020/02/25/tool/gitlab/conclusion_gitlab_ci_cd/}
      \item \url{https://zput.github.io/2020/01/22/tool/gitlab/the_process_structure_ci_cd_base_on_gitlab/})
    \end{itemize}
      %  - 2020-01-22 20:09:00
      %  - 2020-02-25 21:06:00

   \item 构建基于go-micro后台微服务框架。为适应团队新的开发流程,兼容旧服务与小程序,对外提供了grpc与grpc-gw(json)两个服务。在内部, 模块分为API层, work层,core层,内部模块之间交流使用grpc。参与整个框架制定,实现对外提供的两个服务(grpc,grpc-gw)与core层等,开发调研服务周边的设施(log, token, sql driver). 缩短与前端对接时间,定好*.proto协议,即可分发给个模块人员。
      %  - 2020-04-25 22:16:00

   \item  参与公司两条新产品AI课与语文课的开发。领导参与的主要模块包括登录相关,权限相关, 课程,题库,学生排课,作业,公众号提醒与消息推送等,学生人数从0到3.5万人数增长。现在为这两条线的后台项目负责人,开发新需求与维护线上服务稳定。
      %  - 2020-08-15 20:16:00

   \item  负责销售系统,期间开发客户关系维护系统,电话营销智能拨打系统, 提高销售人员工作效率3倍以上.
      %  - 2019-08-15 20:16:00
\end{itemize}

\datedsubsection{\textbf{深圳云之梦科技有限公司}, 研发部开发工程师}{2018.9-2019.7}
\begin{itemize}[parsep=0.2ex]
  \item 开发基础架构平台,主要提供基础数据给小程序,微信公众号,H5。
  \item 构建开放平台,为商户提供统一公司服务的接口。
  \item 为外部商家提供总店,门店的后台管理系统,包括会员注册,衣服商品上下架,提供与线下设备传输交流打通3D试衣等功能。
\end{itemize}

\datedsubsection{\textbf{深圳优科数字化制造有限公司}, 研发工程师}{2016.6-2018.9}
\begin{itemize}[parsep=0.2ex]
  \item 参与工厂改造,设备生产数据采集, 开发工业软件的数据通信,不仅负责与AGV等智能设备的数据收集,还向上提供了工厂看板与移动端小程序的数据展示,较大提高甲方验收工厂数字化转型满意度.
  \item 负责智能咖啡机项目,从手机点餐下单,到订单数据传输到咖啡机机器人,咖啡完成提示,有效结合了工业与信息的结合,在展会中吸引大量潜在客户.
\end{itemize}

\section{一些开源项目}
\begin{itemize}[parsep=0.2ex]
  \item \textbf{innodb\_view}用于分析MySQL的innodb引擎中物理存储文件。(\url{https://github.com/zput/innodb_view})
    \begin{itemize}[parsep=0.2ex]
        \item Innodb\_view是一个直接访问MySQL InnoDB存储引擎文件的Golang实现。通过命令行可以遍历所有已经使用的页,分析每个页的类型; 分析Inode page页面组成;Index page页面的组成等。此外,这个项目可加深对MySQL innodb物理页面内部结构理解。
    \end{itemize}

  \item \textbf{zput\_net\_golang}基于事件驱动(Reactor模式)的高性能,非阻塞和轻量级网络框架,不使用标准golang语言net网络包, 它的多路复用根据不同系统使用不同的系统函数(epoll(linux系统)和kqueue(FreeBSD系统)), 轻松快速搭建高性能服务器。(\url{https://github.com/zput/zput_net_golang})
    \begin{itemize}[parsep=0.2ex]
          \item 非阻塞I/O。
          \item 多Goroutine支持,每个Goroutine运行一个事件驱动的事件循环。
          \item 读写缓冲区使用可伸缩的环形缓冲区。
          \item 支持端口重用(SO\_REUSEPORT)。
          \item 支持事件定时任务。
    \end{itemize}

  \item \textbf{ringbuffer}多功能环形缓存,可配置加锁(线程安全)与不加锁(性能更好),当空间满自动扩展,可预先查看缓存中的数据,探索API的方式读等。(\url{https://github.com/zput/ringbuffer})
    \begin{itemize}[parsep=0.2ex]
          \item 在New构造函数的时候,通过参数决定是加锁(线程安全)还是不加锁。
          \item 当环形缓存空间满后,可以自动扩展内存。
          \item 当使用探索类函数(ExploreBegin()----ExploreRead()/ExploreSize()----ExploreCommit()/ExploreBreak()),可以预先探索缓存中的数据,最后可以决定是提交还是放弃。
    \end{itemize}

  \item \textbf{zxc\_net}zxc\_net是基于Reactor模式的多线程C++网络库。(\url{https://github.com/zput/zxc_net})
    \begin{itemize}[parsep=0.2ex]
          \item 依赖于C++11提供的std::thread库,多线程表现在:
            \begin{itemize}[parsep=0.2ex]
              \item 为accept socket 的线程中的acceptFd 开启SO\_RESUEPORT选项,多个线程拥有自己的Acceptor进行接收客户端连接。
              \item 为每个拥有acceptFd的线程建立了线程池, 便于分发接收到的clientFd。 解决了TCP连接负载均衡的问题, 减少客户连接等待时间。
              \item 非阻塞的poll模式,
            \end{itemize}
          \item 拥有read/write buff缓冲区, readv + LT 触发可以读取较大数据。
    \end{itemize}
\end{itemize}


\section{杂谈}
\begin{itemize}[parsep=0.2ex]
  \item \textbf{golang调度} \url{https://zput.github.io/go-goroutine/}
  \item \textbf{内部做的技术分享PPT}:
  \begin{itemize}[parsep=0.2ex]
    \item \textbf{chan}: \url{https://docs.google.com/presentation/d/1q0NP8jtJacfSjFaKFSTi_5MMXNVuSomji3ytbSLKgts/edit?usp=sharing}
    \item \textbf{goroutine}: \url{https://docs.google.com/presentation/d/1JoRSCB_UVHClv3JpkyvM5CWfu2reFvxAXgD1LgvXCUU/edit?usp=sharing}
  \end{itemize}
\end{itemize}

\section{技能清单}
\begin{itemize}[parsep=0.2ex]
  \item \textbf{编程语言}: Go, C++, C, SQL, Shell
  \item \textbf{数据库相关}: MySQL/PostgreSQL/Redis/MongoDB
  \item \textbf{消息队列}: kafka
  \item \textbf{一些工程构建}: kubernets, docker, helm, gitlab, ci/cd, beego, go-micro, grpc
\end{itemize}


%% Reference
%\newpage
%\bibliographystyle{IEEETran}
%\bibliography{mycite}
\end{document}


\begin{itemize}[parsep=0.2ex]
  \item 开发小程序,微信公众号,淘宝内的H5架构平台。
  \item 为商户提供公司的开放平台接口。
  \item 为外部商家提供总店门店操作, 会员注册, 衣服商品CRUD操作, 提供与线下设备传输交流打通3D试衣等功能等开放平台的API。
  \item 可扩展性的的小程序,H5的公共平台。
\end{itemize}






%Feature:是什么
%Advantage:比别人好在哪些地方
%Benefit:如果雇佣你,招聘方会得到什么好处
%其次,写简历和写议论文不同,过分的论证会显得自夸,反而容易引起反感,所以要点到为止。这里的技巧是,提供论据,把论点留给阅读简历的人自己去得出。放论据要具体,最基本的是要数字化,好的论据要让人印象深刻。
%
%举个例子,下边内容是虚构的:
%
%2006年,我参与了手机XX网发布系统WAPCMS的开发(这部分是大家都会写的)。
%作为核心程序员,我不但完成了网站界面、调度队列的开发工作,更提出了高效的组件级缓存系统,通过碎片化缓冲有效的提升了系统的渲染效率。
%(这部分是很多同学忘掉的,要写出你在这个项目中具体负责的部分,以及你贡献出来的价值。)
%在该系统上线后,Web前端性能从10QPS提升到200QPS,服务器由10台减少到3台(通过量化的数字来增强可信度)。
%2008年我升任WAPCMS项目负责人,带领一个3人小组支持着每天超过2亿的PV(这就是Benefit。你能带给前雇主的价值,也就是你能带给新雇主的价值。)。
%
%有同学问,如果我在项目里边没有那么显赫的成绩可以说怎么办?讲不出成绩时,就讲你的成长。因为学习能力也是每家公司都看中的东西。你可以写你在这个项目里边遇到了一个什么样的问题,别人怎么解决的,你怎么解决的,你的方案好在什么地方,最终这个方案的效果如何。
%
%具体、量化、有说服力,是技术简历特别需要注重的地方。
%
%(以上内容在写完简历后,对每一段进行评估,完成后再删除)









%- 企业上云:整合服务, 整理路由,上k8s。
%当时有十几个beego独立的微服务,前端调用的相同前缀URI但不同后缀的接口,可能分布在几个微服务中,
%无法进行负载均衡,分散流量,作为发起者,与前端商讨路由整合重构后缩减微服务到7个,并编写helm,服务都上到k8s,(\url(helm参考))
%性能从10QPS提升到200QPS,测试加正式服务器由10台减少到5台。
%  - 2019-10-24 22:03:00
%
%- 规划编写服务CI/CD基于gitlab自动发布k8s流程。
%其中包括。gitlab-ci。yml的编写,gitlab-runner部署以及它默认docker等(\url(https://zput.github.io/2020/01/22/tool/gitlab/learn_gitlab/))。
%当自动发布上线后,开发打包部署到交给测试人员测试,从至少15分钟,降低到5分钟内。
%  - 2020-01-22 20:09:00
%  - 2020-02-25 21:06:00
%
%- 构建基于go-micro后台微服务框架
%适应团队新的开发流程,和兼容以前的服务与小程序,对外提供了grp与grpc-gw(json)两个服务,内部模块分为API层(), work层(),core层(),内部服务交流直接使用的go-micro自带提供的
%参与整个框架制定,实现对外提供的两个服务(grpc,grpc-gw)与core层等,开发调研了服务周边的设施(log, token, sql driver)
%缩短与前端对接时间,定好*。proto协议,即可分发给个模块人员。
%
%  - 2020-04-25 22:16:00
%- 参与公司两条新产品AI线,语文课的开发,
%领导参与的主要模块包括登录相关,权限相关
%课程,题库,学生排课,作业,公众号提醒与消息推送等
%学生人数从0到1。5万人数增长。
%现在为这两条线的后台项目负责人,开发新需求与维护线上服务稳定。
%  - 2020-08-15 20:16:00
























%token --> rbac --> grpcgw
%- kubernets log/monitory
%//- logrus
%- https://github。com/lib/pq
%- 构建基于go-micro后台微服务框架
%  - blog





% network page 
% https://www.overleaf.com/project





















