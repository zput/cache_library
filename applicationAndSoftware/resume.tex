% !TEX TS-program = xelatex
% !TEX encoding = UTF-8 Unicode
% !Mode:: "TeX:UTF-8"

\documentclass{resume}
\usepackage{zh_CN-Adobefonts_external} % Simplified Chinese Support using external fonts (./fonts/zh_CN-Adobe/)
%\usepackage{zh_CN-Adobefonts_internal} % Simplified Chinese Support using system fonts
\usepackage{linespacing_fix} % disable extra space before next section
\usepackage{cite}
\usepackage{hyperref}

\begin{document}
\pagenumbering{gobble} % suppress displaying page number

\name{golang后台研发-赵轩超}

% {E-mail}{mobilephone}{homepage}
% be careful of _ in emaill address
\contactInfo{(+86) 177-1164-4027}{ffzxc.do@gmail.com}{GitHub @zput}{}
% {E-mail}{mobilephone}
% keep the last empty braces!
%\contactInfo{xxx@yuanbin.me}{(+86) 131-221-87xxx}{}
 
\section{个人信息}
\begin{itemize}[parsep=0.2ex]
  \item 赵轩超/男/1993-12
  \item 本科/湖南理工学院/自动化(2012.09-2016.06)
  \item 技术博客:zput.github.io
  \item Github:http://github.zput
\end{itemize}

\section{工作经历}
\datedsubsection{\textbf{深圳云之梦科技有限公司}, 平台架构部golang开发工程师}{2018.9-2019.7}
\begin{itemize}[parsep=0.2ex]
  \item 开发小程序,微信公众号,淘宝内的H5架构平台.
  \item 为商户提供公司的开放平台接口.
  \item 为外部商家提供总店门店操作, 会员注册, 衣服商品CRUD操作, 提供与线下设备传输交流打通3D试衣等功能等开放平台的API.
  \item 可扩展性的的小程序,H5的公共平台.
\end{itemize}

\datedsubsection{\textbf{深圳优科数字化制造有限公司}, 研发工程师}{2016.6-2018.9}
\begin{itemize}[parsep=0.2ex]
  \item 与前端或微信小程序连接的多线程后端C++开发, 工业软件的数据通信的开发, 与AGV等智能设备数据连接的开发.
\end{itemize}

\section{一些开源项目}
\begin{itemize}[parsep=0.2ex]
  \item \textbf{innodb\_view}用于分析MySQL的innodb引擎中物理存储文件.(\url{https://github.com/zput/innodb\_view})
    \begin{itemize}[parsep=0.2ex]
        \item TODO
    \end{itemize}

  \item \textbf{zput\_net\_golang}基于事件驱动(Reactor模式)的高性能,非阻塞和轻量级网络框架,不使用标准golang语言net网络包, 它的多路复用根据不同系统使用不同的系统函数(epoll(linux系统)和kqueue(FreeBSD系统)), 轻松快速搭建高性能服务器.(\url{https://github.com/zput/zput\_net\_golang})
    \begin{itemize}[parsep=0.2ex]
          \item 非阻塞I/O。
          \item 多Goroutine支持,每个Goroutine运行一个事件驱动的事件循环。
          \item 读写缓冲区使用可伸缩的环形缓冲区。
          \item 支持端口重用(SO\_REUSEPORT)。
          \item 支持事件定时任务。
    \end{itemize}

  \item \textbf{ringbuffer}多功能环形缓存,可配置加锁(线程安全)与不加锁(性能更好),当空间满自动扩展,可预先查看缓存中的数据,探索API的方式读等.(\url{https://github.com/zput/ringbuffer})
    \begin{itemize}[parsep=0.2ex]
          \item 在New构造函数的时候,通过参数决定是加锁(线程安全)还是不加锁。
          \item 当环形缓存空间满后,可以自动扩展内存。
          \item 当使用探索类函数(ExploreBegin()----ExploreRead()/ExploreSize()----ExploreCommit()/ExploreBreak()),可以预先探索缓存中的数据,最后可以决定是提交还是放弃。
    \end{itemize}

  \item \textbf{zxc\_net}zxc\_net是基于Reactor模式的多线程C++网络库.(\url{https://github.com/zput/zxc\_net})
    \begin{itemize}[parsep=0.2ex]
          \item 依赖于C++11提供的std::thread库,多线程表现在:
            \begin{itemize}[parsep=0.2ex]
              \item 为accept socket 的线程中的acceptFd 开启SO\_RESUEPORT选项,多个线程拥有自己的Acceptor进行接收客户端连接.
              \item 为每个拥有acceptFd的线程建立了线程池, 便于分发接收到的clientFd. 解决了TCP连接负载均衡的问题, 减少客户连接等待时间.
              \item 非阻塞的poll模式,
            \end{itemize}
          \item 拥有read/write buff缓冲区, readv + LT 触发可以读取较大数据.
    \end{itemize}
\end{itemize}


\section{杂谈}
\begin{itemize}[parsep=0.2ex]
  \item \textbf{golang调度} \url{https://zput.github.io/categories/golang%E8%B0%83%E5%BA%A6/}
  \item \textbf{内部做的技术分享PPT}:
  \begin{itemize}[parsep=0.2ex]
    \item \textbf{chan}: \url{https://docs.google.com/presentation/d/1_MHlcc7JOHVGNGYSUVtMDuKA_rKWi76IU9OEkb4XziY/edit?usp=sharing}
    \item \textbf{goroutine}: \url{https://docs.google.com/presentation/d/1_MHlcc7JOHVGNGYSUVtMDuKA_rKWi76IU9OEkb4XziY/edit?usp=sharing}
  \end{itemize}
\end{itemize}

\section{技能清单}
\begin{itemize}[parsep=0.2ex]
  \item \textbf{编程语言}: Go, C++, C, SQL, Shell
  \item \textbf{数据库相关}: MySQL/PostgreSQL/Redis/MongoDB
  \item \textbf{消息队列}: kafka
  \item \textbf{一些工程构建}: kubernets, docker, helm, gitlab, ci/cd, beego, go-micro, grpc
\end{itemize}


%% Reference
%\newpage
%\bibliographystyle{IEEETran}
%\bibliography{mycite}
\end{document}
